\section{Conclusion}

In this work we have introduced VQ-VAE, a new family of models that combine VAEs with vector quantisation to obtain a discrete latent representation. We have shown that VQ-VAEs are capable of modelling very long term dependencies through their compressed discrete latent space which we have demonstrated by generating $128\times128$ colour images, sampling action conditional video sequences and finally using audio where even an unconditional model can generate surprisingly meaningful chunks of speech and doing speaker conversion. All these experiments demonstrated that the discrete latent space learnt by VQ-VAEs capture important features of the data in a completely unsupervised manner. Moreover, VQ-VAEs achieve likelihoods that are almost as good as their continuous latent variable counterparts on CIFAR10 data. We believe that this is the first discrete latent variable model that can successfully model long range sequences and fully unsupervisedly learn high-level speech descriptors  that are closely related to phonemes.